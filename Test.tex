\documentclass[Letterpaper, 12pt]{article}
%\usepackage[utf8]{inputenc}
\usepackage{amsmath, fancyhdr}
\usepackage[margin=1in]{geometry}
%\parskip = 0.2in

\usepackage{enumitem}
\setlist[enumerate, 1]{
label = (\alph*),
itemsep = 6\baselineskip
}

\title{Weekly}
\author{wenjuan.li }
\date{October 2022}
\begin{document}
%-------------------------------
%	HEADERS and FOOTERS SECTION
%-------------------------------
%\maketitle
\pagestyle{fancy}
%... then configure it.
\fancyhead{} % clear all header fields
%\fancyhead[C]{\textbf{: Weekly Review 1 for MTH130 (Instructor: Dr. Wenjuan Li}}
\fancyhead[L]{\textbf{MTH130 Weekly Review}}
\fancyhead[R]{\textbf{Name:}\hspace{1.5in}}
%\fancyfoot{} % clear all footer fields
%\fancyfoot[LE,RO]{\thepage}
%\fancyfoot[LO,CE]{From: K. Grant}
%\fancyfoot[CO,RE]{To: Dean A. Smith}

\begin{center} {\large\textbf{Transformations of Functions}}\\
	\smallskip
	\textbf{Due Date: 10/12/2022}
\end{center}
%\noindent\textbf{Name:} \underline{\hspace{6cm}} \\

%-------------------------------
%	Content SECTION
%-------------------------------
\noindent \textbf{Ex.1 Fill in the blanks.} \\
%\vspace{0.25in}
%-------------------------------
%	Table SECTION
%-------------------------------
\begingroup
\setlength{\tabcolsep}{10pt} % Default value: 6pt
\renewcommand{\arraystretch}{1.5} % Default value: 1
\begin{table}[htbp]
	%\caption{Transformations of Functions}
	%\bigskip
	\centering
	\begin{tabular}{|c|c|c|}
		\hline
		Function Notation        & Change in Coordinate Point            & \hspace{1cm} Transformation \hspace{1cm} \\ [2ex] % inserts table %heading
		\hline
		$y=f(x)+c, c>0$          & $(x, y) \rightarrow(x, y+c)$          &                                          \\ \hline
		$y=f(x)-c, c>0$          & $(x, y) \rightarrow(x, y-c)$          &                                          \\ \hline
		$y=f(x+c), c>0$          & $(x, y) \rightarrow(x-c, y)$          &                                          \\ \hline
		$y=f(x-c), c>0$          & $(x, y) \rightarrow(x+c, y)$          &                                          \\ \hline
		$y=f(kx), k>1$           & $(x, y) \rightarrow(\frac{1}{k}x, y)$ &                                          \\ \hline
		$y=f(\frac{1}{k}x), k>1$ & $(x, y) \rightarrow(kx, y)$           &                                          \\ \hline
		$y=kf(x), k>1$           & $(x, y) \rightarrow(x, ky)$           &                                          \\ \hline
		$y=\frac{1}{k}f(x), k>1$ & $(x, y) \rightarrow(x, \frac{1}{k}y)$ &                                          \\ \hline
		$y=-f(x)$                & $(x, y) \rightarrow(x, -y)$           & \hfill                                   \\ \hline
		$y=f(-x)$                & $(x, y) \rightarrow(-x, y)$           & \hfill                                   \\ \hline
	\end{tabular}
	\label{table:nonlin}
\end{table}
\endgroup

%-------------------------------
%	Practice Problems
%-------------------------------
\noindent \textbf{Ex.2 Determine how the graphs below can be obtained from the graph of $f$.}

\begin{enumerate}[itemsep = 2\baselineskip]
	\item $y=f(\frac{1}{3} x) +6$
	      \vspace*{3\baselineskip}
	\item $y=f(\frac{1}{3} x) +6$
	\item $y=-f(2x)+7$
\end{enumerate}


\end{document}



